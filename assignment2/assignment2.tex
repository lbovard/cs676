\documentclass[10pt,english]{article}
\usepackage[T1]{fontenc}
\usepackage[latin9]{inputenc}
\usepackage[letterpaper]{geometry}
\geometry{verbose,bmargin=3cm,lmargin=2.5cm,rmargin=3cm}
\usepackage{amsthm}
\usepackage{amsmath}
\usepackage{amsfonts}
\usepackage{amssymb}
\usepackage{fancyhdr}
\usepackage{graphicx}
\theoremstyle{plain}
\newcommand{\dt}{\Delta t}
\pagestyle{fancy}
\fancyhead[L]{Luke Bovard - 20288133}
\fancyhead[c]{CS 676 - Assignment 2}
\fancyhead[R]{\today}
\usepackage{babel}
\begin{document}
\begin{enumerate}
%%%%%%%%%%%%%%%%%%%%
%QUESTION 1
%%%%%%%%%%%%%%%%%%%%
\item 
\begin{enumerate}
\item See the file \texttt{binomialDelta.m} attached.

\item See the file \texttt{interpDelta.m} attached. It is important to note that here I do not use the built-in MATLAB function \texttt{interp1} as profiling the code for this question showed that this function was taking up all of the CPU time. I think this is  because \texttt{interp1} computes the linear approximations for all points and then at the end choses the one you want. Instead I just implemented the linear interpolation myself and the code sped up significantly. The MATLAB code profiler showed that what previously took 5 minutes now only took 40 seconds.  The code is also fairly fast as it exploits vectorisation on logical operations.

\item Figure \ref{q1} shows the exact vs approximate solution of delta hedging as $t_{N-1}$. As we can see the approximate solution is pretty good. Let us first comment on the shape. This is exactly the shape one would expect. Consider the option value at expiry. It is simply $V=\max(S-K,0)$. Recall that $\alpha = \partial V/\partial S$ so in this case we would obtain the function $\alpha = 0$ if $ S<K$ and $\alpha=1$ if $ S>K$. This is nothing but the Heaviside step function. Now at a time-step before this time we except the $\alpha$ curve to very close to a step function, which as we can see, it is.

Now when we consider the exact solution we have a very smooth curve while for the approximate we have a more rigid curve. This is because finite difference approximations do not deal well with discontinuities. The exact solution has an analytical form which is smooth at $t_{N-1}$ and hence the derivative is also smooth. Even though this function is smooth, it is still approaching a vertical line which would be a discontinuity. The analytical solution doesn't care about this but the finite difference approximate solution does and hence will have a difficult time approximating a discontinuity. To resolve this one should probably use a finite volume techniques which are developed to better handle discontinuities.
\begin{figure}
\begin{center}
\includegraphics[scale=0.7]{q1.png}
\caption{Delta hedge position $\alpha$ as a function of strike price $S$ at $t_{N-1}$.}
\label{q1}
\end{center}
\end{figure}
\end{enumerate}

%%%%%%%%%%%%%%%%%%%%
%QUESTION 2
%%%%%%%%%%%%%%%%%%%%
\item
\begin{enumerate}
%%part a
\item The code is implemented in \texttt{q2.m}. Some comments are necessary on the code. There are two places where vectorisation was unsuccessful. The first is calling the \texttt{interpDelta} function which could not be vectorised because MATLAB complained that the input arrays are of different sizes. This is not a problem because the function, as discussed in Question 1, has been vectorised internally so it is very fast. Secondly doing the rebalancing of the bank could be vectorised since it is a recursive relationship, however it is an inhomogeneous relationship which, while having an exact solution (which is provided by commented out) is inefficient to compute. It is far faster to use a simple loop to calculate the recursion. The code has been profiled and the effects of the bank updating are minimal so using the exact solution might cause a miniscule speed up that would be unnoticable.  
\begin{figure}
\begin{center}
\includegraphics[scale=0.7]{q2no_hedge.png}
\caption{Probability density of Profit and Loss with no delta hedging.}
\label{nohedge}
\end{center}
\end{figure}
\begin{figure}
\begin{center}
\includegraphics[scale=0.7]{q2n_12.png}
\caption{Probability density of Profit and Loss delta hedging monthly.}
\label{monthly}
\end{center}
\end{figure}
\begin{figure}
\begin{center}
\includegraphics[scale=0.7]{q2n_52.png}
\caption{Probability density of Profit and Loss delta hedging weekly.} 
\label{weekly}
\end{center}
\end{figure}
\begin{figure}
\begin{center}
\includegraphics[scale=0.7]{q2n_250.png}
\caption{Probability density of Profit and Loss delta hedging daily.}
\label{daily}
\end{center}
\end{figure}

Figures \ref{nohedge}-\ref{daily} show the probability densities of various hedging frequenicies. Note that each figure has different axes and that here a negative value of profit and loss refers to a profit while a positive value refers to a loss. Figure \ref{nohedge} has no-hedging and we can see that with no hedging there is an almost $25\%$ chance of losing money on this portfolio. This makes sense because when we do no hedging the value of the portfolio at expiry is simply
\begin{align*}
\Pi(T) = -V(S(T),T)+ B(t_{N-1})e^{r\Delta t}
\end{align*}
but since the recursive formula for the bank is simply $B(t_{i})=e^{r\Delta t}B(t_{i-1}), B(0)=V(0)$ this formula becomes
\begin{align*}
\Pi(T) = -V(S(T),T)+ B(0)e^{rT} = -\max(S-K,0)+V(0)e^{rT}
\end{align*}
so if the asset has $S<K$ then the portfolio is simply worth $\Pi=V(0)e^{rT}$ a constant. In fact the profit and loss formula now becomes
\begin{align*}
\text{P\&L}= \frac{e^{-rT}\Pi(T)}{V(0)}= \frac{V(0)e^{rT}e^{-rT}}{V(0)} = 1
\end{align*}
so the P\&L is exactly 1 which, looking at Figure \ref{nohedge} we see that is exactly where the spike is located. The reason the probability is large is because this is the value of the portfolio for all $S<K$ which covers a wide range of values. Now as $S>K$ the value of the porfolio becomes more negative (i.e. has a greater profit) we see that the potential profit decreases. It turns out in this simulation there was an observed profit of $-18$ but the probability is incredibly tiny.

Figures \ref{monthly}-\ref{daily} show an almost normal distribution of the profit and loss. Looking at Figure \ref{monthly} we see that the probability distribution is very broad with values ranging from $-0.9$ to $0.5$. When we switch to rebalancing weekly we see that the distribution is becoming less broad and thes values only range from about $-0.3$ to $0.3$. When we rebalance daily see that the distribution is becoming narrower and the profit and loses range from $-0.15$ to $0.1$. Additionally the probability of having $0$ profit increases. This makes sense. In an ideal world, we can rebalance continuously and we can always have a risk-free portfolio. But we cannot do this in the real world so we must rebalance at discrete times. As we delta hedge more frequently we are better able to hedge away the risk and the probability of a risk-free portfolio increases to $1$. This explains why when we delta hedge more frequently the width of the distribution becomes narrower and narrower. If we delta hedged more frequently then we would see an even narrower peak, but in reality delta hedging is not something that can be done for free as there are transaction costs to consider. Thus a more accurate hedging model would include these transaction costs.
%%part b
\item This is implemented in \texttt{dVaRCVaR.m} and is attached. There is little to say here as it is a direct translation of the formulas provided.

%%part c
\item Code for this is also implemented in \texttt{q2.m}. Table 1 contains the data for the 4 rebalancing options. As can be seen, this table gives a more quantitative assessment of the figures. Looking at the mean and standard deviation, we can easily see that the more frequent we delta hedge the probability curve gets narrower and narrower with the mean very close to zero (a riskless portfolio) and small standard deviations. If we do not delta hedge, we see that the standard deviation and mean of that case do not give a clear picture of what is going on unlike the other cases which are approximately normal and can be described accurately with the mean and std. Instead this is where the VaR and CVaR quantities are useful in telling us a bit about the distribution. The VaR and CVaR give a slightly more meaningful assessment of the risk. Let us consider the representative case of rebalancing monthly. The standard deviation is $\sigma=0.1374$ while the VaR and CVaR give $0.2222,0.3113$ respectively. Looking at Figure \ref{monthly} we see that outside of one standard deviation, there is still a significant tail and the VaR and CVaR capture more of that tail.

The situation is even worse with no hedging as the distribution is far from normal so the standard deviation cannot be interpretted as nicely as with the normal while the VaR and CVaR give us a more reasonable assessment of the portfolio. It is worth pointing out that this portfolio could never exceed a loss of 1 while the VaR and CVaR suggest it could. Given that these numbers are roughly 2 and 3 times bigger, they suggest a significantly larger risk in this porfolio. Indeed the probability density distribution suggests that a loss of 1 is significantly more likely than other possibilities (by almost double).
\begin{table}[t]

\centering
\begin{tabular}{|c|c|c|c|c|}
\multicolumn{5}{c}{Call Option}\\
\hline
	$N$ & Mean & Std & VaR(95\%) & CVaR(95\%)  \\
\hline
	None & 0.5838 & 1.718 & 3.969 & 5.376 \\
	12	&	-0.001195 & 0.1374 & 0.2222 & 0.3113 \\
	52	&	 0.001377 & 0.0665 & 0.1088 & 0.1532 \\
	250	&		0.00000812 & 0.0297 & 0.0482 & 0.0685\\
\hline
\end{tabular}
\caption{The mean, std, VaR, CVaR for four types of rebalancing of delta hedging. For VaR and CVaR the percentage is $\beta=95\%$.}
\end{table}
\end{enumerate}

%%%%%%%%%%%%%%%%%%%%
%QUESTION 3
%%%%%%%%%%%%%%%%%%%%
\item Assume that the volatility follows a CIR process 
\begin{align*}
dv = -\lambda (v-\bar{v})dt + \eta \sqrt{v}dZ^{(2)},
\end{align*}
and the Milstein scheme is given as
\begin{align*}
v_{n+1} = v_{n} + a\Delta t + b \Delta W^{(2)} + \frac{1}{2}bb'((\Delta W^{(2)})^{2}-\Delta t).
\end{align*}
Here $a=-\lambda(v-\bar{v}),b=\eta \sqrt{v},$ and $b'=\eta/2\sqrt{v}$. This implies that $bb'/2 = \eta^{2}/4$. Plugging these in yields
\begin{align*}
v_{n+1} &= v_{n}-\lambda(v_{n}-\bar{v})\Delta t + \frac{\eta^{2}}{4}((\Delta W^{(2)})^{2}-\Delta t) + \eta \sqrt{v_{n}}\Delta W^{(2)}\\
&= -\lambda(v_{n}-\bar{v})\Delta t - \frac{\eta^{2}}{4}\Delta t + v_{n} + \sqrt{v_{n}}\eta \Delta W^{(2)}+ \frac{\eta^{2}}{4}(\Delta W^{(2)})^{2}.
\end{align*}
We now observe that $v_{n}+\eta\sqrt{v_{n}}\Delta W^{(2)} + \eta^{2}/4 (\Delta W^{(2)})^{2} = (\sqrt{v_{n}}+\eta/2 \Delta W^{(2)})^{2}$. Thus we immediately have
\begin{align*}
v_{n+1} = \left(\sqrt{v_{i}}+\frac{\eta}{2}\sqrt{\Delta t}\phi_{i}^{(2)}\right)^{2} - \lambda(v_{i}-\bar{v})\Delta t - \frac{\eta^{2}}{4}\Delta t,
\end{align*}
where we recall that $\Delta W^{(2)} = \sqrt{\Delta t}\phi^{(2)}_{i}$.

Now assume that $v_{i}=0$. We have that the squared term is always positive so to ensure that $v_{i+1}>0$ we must have that
\begin{align*}
\lambda\bar{v}\Delta t -\frac{\eta^{2}}{4}\Delta t > 0 \Rightarrow \lambda \bar{v}-\frac{\eta^{2}}{4} > 0\Rightarrow \frac{4\lambda\bar{v}}{\eta^{2}}>1
\end{align*}
as required.
%%%%%%%%%%%%%%%%%%%%
%QUESTION 4
%%%%%%%%%%%%%%%%%%%%
\item
\begin{enumerate}
\item The code is implemented in \texttt{hesvol.m} and \texttt{q4b.m} and is attached. We find that the initial option value is $V_{0}=4.322519$. As will be discussed in question 6 this answer is about $0.3\%$ off the exact.

\item Figures \ref{hesstrike} shows the implied volatility as a function of strike price under Heston's model. A problem with the Black-Scholes model is that when real data is input and the implied volatility is plotted we see a smile, despite the assumption that $\sigma$ is constant. In Heston's model we see that the implied volatility is in fact a smile, so this model provides a theoretical model for stock evolution that has volatility smile built into it. Thus when it comes to real-world modelling Heston's model might provide a better option price than the Black-Scholes.

If we look at Figure \ref{hesexpiry} we see that the implied volatility is roughly constant regardless of the expiry time.  This result states that with the same strike and initial price but with different expiry times the implied volatility from the initial option price is the same. The Black-Scholes equation would give a straight line because the volatility is constant in that model. Thus we see that Heston's model is able to predict a volatility smile and retain some of the features of the Black-Scholes model, such as constant implied volatility from initial option prices independent of the expiry time.
\begin{figure}
\begin{center}
\includegraphics[scale=0.7]{q4b_strike.png}
\caption{The implied volatility as a function of strike price under Heston's stochastic volatility model. The volatility smile can be seen.}
\label{hesstrike}
\end{center}
\end{figure}
\begin{figure}
\begin{center}
\includegraphics[scale=0.7]{q4b_expiry.png}
\caption{The implied volatility as a function of expiry price under Heston's stochastic volatility model.}
\label{hesexpiry}
\end{center}
\end{figure}
\end{enumerate}
%%%%%%%%%%%%%%%%%%%%
%QUESTION 5
%%%%%%%%%%%%%%%%%%%%
\item
\begin{enumerate}
\item Let $X_{t}=(1+Z_{t})e^{t^{2}/2}$. Let $Z_{t}$ be a standard Browian motion so if we set $X_{t}=Z_{t}$ then $dX_{t} = dZ_{t}$ is an Ito process with $a=0$ and $b=1$. Now consider $g(x,t)=(1+x)e^{t^{2}/2}$. We have that
\begin{align*}
\frac{\partial g}{\partial t} = tg \qquad \frac{\partial g}{\partial x}= e^{t^{2}/2} \qquad \frac{\partial^{2}g}{\partial x^{2}} = 0
\end{align*}
Set $Y_{t}=g(X_{t},t)$ so by Ito's lemma we have that
\begin{align*}
dY_{t} &= \left(\frac{\partial g}{\partial t} + a\frac{\partial g}{\partial x} + \frac{1}{2}b^{2}\frac{\partial^{2}g}{\partial x^{2}}\right)dt+b\frac{\partial g}{\partial x}dZ_{t}\\
&= tg dt + e^{t^{2}/2}dZ_{t}\\
&= tY_{t} dt + e^{t^{2}/2}dZ_{t}
\end{align*}
as required.

\item Let $X_{t}=Z_{t}^{3}-3tZ_{t}$. As before $Z_{t}$ is a standard Browian motion so if we set $X_{t}=Z_{t}$ then $dX_{t}=dZ_{t}$ is an Ito process with $a=0$ and $b=1$. Now consider $g(x,t)=x^{3}-3tx$. 
\begin{align*}
\frac{\partial g}{\partial t} = -3x \qquad \frac{\partial g}{\partial x}= 3x^{2}-3t \qquad \frac{\partial^{2}g}{\partial x^{2}}= 6x
\end{align*}
Set $Y_{t}=g(X_{t},t)$ so by Ito's lemma we have that
\begin{align*}
dY_{t} &= \left(\frac{\partial g}{\partial t} + a\frac{\partial g}{\partial x} + \frac{1}{2}b^{2}\frac{\partial^{2}g}{\partial x^{2}}\right)dt+b\frac{\partial g}{\partial x}dZ_{t}\\
&= (-3X_{t}+3X_{t})dt + (3X_{t}^{2}-3t)dZ_{t}\\
&= (3Z_{t}^{2}-3t)dZ_{t}
\end{align*}
since $X_{t}=Z_{t}$. This gives the required result.
\end{enumerate}

%%%%%%%%%%%%%%%%%%%%
%QUESTION 6
%%%%%%%%%%%%%%%%%%%%
\item The implementation of the exact solution from \cite{heston} is provided in \texttt{q6.m} and \texttt{charfunc.m} which are attached. As discussed in the code, but repeated here, the integral ranges from $0,\infty$ however the function \texttt{charfunc} is unable to handle really large values of $\phi$ so the integral is truncated at $5000$. Since the characteristic function decays very quickly (both due to the presence of $e^{-i\phi}$ and $1/\phi$ type terms) the error in this is very tiny and testing has shown that the difference between an upper bound of $1000$ and $5000$ is smaller than $10^{-15}$. The lower bound is replaced by $10^{-10}$ instead of $0$ since this implementation is naive and does not fix the divison by zero at $\phi=0$. However, as can easily be seen $\lim_{\phi\rightarrow 0}$ exists so the integral is well-defined. 

Using the data from question 4, the exact solution gives $V_{0}=4.322519$ while the Heston model from question 4 gives $V_{0}=4.337212$ which gives a relative error of $0.33\%$ which is quite good. It is clear that when an exact formula exists, it is a good idea to use it since the running time of the exact solution is under $0.01$ seconds.
\end{enumerate}
\begin{thebibliography}{1}
\bibitem{heston} S. ~Heston,\textit{A closed-form solution for options with stochastic volatility with applications to bond and currency options}, Review of Financial Studies 6, 327-343 (1993)
\end{thebibliography}

\end{document} 


