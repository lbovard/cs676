\documentclass[10pt,english]{article}
\usepackage[T1]{fontenc}
\usepackage[latin9]{inputenc}
\usepackage[letterpaper]{geometry}
\geometry{verbose,bmargin=3cm,lmargin=2.5cm,rmargin=3cm}
\usepackage{amsthm}
\usepackage{amsmath}
\usepackage{amsfonts}
\usepackage{amssymb}
\usepackage{fancyhdr}
\usepackage{graphicx}
\theoremstyle{plain}
\newcommand{\dt}{\Delta t}
\pagestyle{fancy}
\fancyhead[L]{Luke Bovard - 20288133}
\fancyhead[c]{CS 676 - Assignment 3}
\fancyhead[R]{\today}
\usepackage{babel}
\begin{document}
\begin{enumerate}
%%%%%%%%%%%%%%%%%%%%
%QUESTION 1
%%%%%%%%%%%%%%%%%%%%
\item The code is implemented in files \texttt{q1.m} and \texttt{fd\_european.m} and comments are therein.
\begin{table}[t]
\centering
\begin{tabular}{|c|c|c|}
\multicolumn{3}{c}{Put Option}\\
\hline
	Implicit & CN & Rannacher\\
\hline
2.3851 & 2.0360 & 3.9989\\
2.2198 & 2.5249 & 3.9978\\
2.1172 & 2.3032 & 3.9988\\
2.0607 & 2.1642 & 3.9994\\
2.0309 & 2.0856 & 3.9996\\
2.0156 & 2.0438 & 4.0010\\
2.0079 & 2.0223 & 4.0286\\
\hline
\end{tabular}
\caption{The convergence ratio of implicit, Crank-Nicolson, and Rannacher time-stepping schemes.}
\end{table}
%%%%%%%%%%%%%%%%%%%%
%QUESTION 2
%%%%%%%%%%%%%%%%%%%%
\item


%%%%%%%%%%%%%%%%%%%%
%QUESTION 3
%%%%%%%%%%%%%%%%%%%%
\item
%%%%%%%%%%%%%%%%%%%%
%QUESTION 4
%%%%%%%%%%%%%%%%%%%%
\item
%%%%%%%%%%%%%%%%%%%%
%QUESTION 5
%%%%%%%%%%%%%%%%%%%%
\item

%%%%%%%%%%%%%%%%%%%%
%QUESTION 6
%%%%%%%%%%%%%%%%%%%%
\item

\end{enumerate}
\end{document}
